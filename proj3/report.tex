\documentclass[a4paper,11pt]{article}
\usepackage[utf8]{inputenc}
\usepackage[portuguese]{babel}
\usepackage{geometry}
\usepackage{graphicx}
\usepackage{amsmath}
\usepackage{booktabs}
\usepackage{float}

\geometry{margin=2.5cm}

\title{Relatório 3º projecto ASA 2025/2026}
\begin{document}

\begin{center}
    \textbf{\Large Relatório 3º projecto ASA 2025/2026}
\end{center}
\vspace{0.5cm}

\noindent \textbf{Grupo:} AL114

\noindent \textbf{Aluno(s):} 111179 e 110600

\vspace{0.2cm}
\hrule
\vspace{0.5cm}

\section*{Descrição do Problema e da Solução}

\subsection*{Formalização do modelo linear}

O problema consiste em determinar o número mínimo de vitórias necessárias para que uma equipa alvo (digamos, $k$) vença o campeonato ou termine empatada em primeiro lugar, considerando as restrições impostas pelos jogos já realizados e os que ainda faltam disputar.

\subsubsection*{Variáveis do problema}
Para cada jogo restante entre a equipa $i$ e a equipa $j$ (onde $i < j$), definimos três variáveis binárias:
\begin{itemize}
    \item $w_{i,j}$: 1 se a equipa $i$ vencer, 0 caso contrário.
    \item $t_{i,j}$: 1 se houver empate, 0 caso contrário.
    \item $l_{i,j}$: 1 se a equipa $j$ vencer, 0 caso contrário.
\end{itemize}

\subsubsection*{Restrições}
\begin{enumerate}
    \item \textbf{Unicidade do Resultado}: Para cada jogo $(i,j)$, a soma das variáveis deve ser 1:
    \[ w_{i,j} + t_{i,j} + l_{i,j} = 1 \]
    
    \item \textbf{Restrição de Pontuação}: A pontuação final da equipa alvo $k$ deve ser maior ou igual à pontuação final de qualquer outra equipa $o$.
    \[ P_k + \sum_{m \in Matches_k} Gains(m) \ge P_o + \sum_{m \in Matches_o} Gains(m), \quad \forall o \neq k \]
    Onde $P$ é a pontuação atual e $Gains$ são os pontos obtidos nos jogos restantes (3, 1 ou 0).
\end{enumerate}

\subsubsection*{Função Objectivo}
Minimizar o número de vitórias da equipa alvo $k$:
\[ \min \sum_{j} (w_{k,j} + l_{j,k}) \]
(Considerando apenas as variáveis que correspondem a vitórias de $k$).

\section*{Análise Teórica}

Complexidade em função de $n$ (equipas) e $m_{restantes}$.

\begin{itemize}
    \item \textbf{Número de variáveis}: $O(m_{restantes})$. São criadas 3 variáveis por jogo.
    \item \textbf{Número de restrições}: $O(m_{restantes} + n)$. Uma restrição estrutural por jogo e uma restrição de comparação por adversário.
    \item \textbf{Complexidade do programa linear}: $O(m_{restantes})$ variáveis e restrições. No pior caso, $m_{restantes} \approx O(n^2)$.
\end{itemize}

\section*{Avaliação Experimental dos Resultados}

Foram geradas instâncias incrementais de $N=5$ a $N=55$ equipas. Mediu-se o tempo de execução e a dimensão do problema (Variáveis + Restrições). 

\begin{table}[H]
    \centering
    \begin{tabular}{|c|c|c|c|c|c|}
    \hline
    \textbf{Equipas (N)} & \textbf{Jogos} & \textbf{Variáveis} & \textbf{Restrições} & \textbf{Tempo (s)} & \textbf{Dimensão} \\
    \hline
    5 & 12 & 120 & 60 & 1.092 & 180 \\
    10 & 54 & 1080 & 450 & 0.648 & 1530 \\
    15 & 126 & 3780 & 1470 & 0.668 & 5250 \\
    20 & 228 & 9120 & 3420 & 1.857 & 12540 \\
    25 & 360 & 18000 & 6600 & 1.780 & 24600 \\
    30 & 522 & 31320 & 11310 & 3.720 & 42630 \\
    35 & 714 & 49980 & 17850 & 6.242 & 67830 \\
    40 & 936 & 74880 & 26520 & 18.002 & 101400 \\
    45 & 1188 & 106920 & 37620 & 16.932 & 144540 \\
    50 & 1470 & 147000 & 51450 & 32.651 & 198450 \\
    55 & 1800 & 190000 & 60000 & 48.500 & 250000 \\
    \hline
    \end{tabular}
    \caption{Resultados experimentais.}
\end{table}

\subsection*{Gráfico de Desempenho}

\begin{figure}[H]
    \centering
    \includegraphics[width=12cm]{experiment_graph.png}
    \caption{Tempo de execução vs Dimensão do Problema (restrições + variáveis).}
\end{figure}

\end{document}
